%%%%% HTDP

\begin{frame}

\Large \centering
others criticised the content
\end{frame}


\begin{frame}[fragile]

\begin{center}
{\bf Criticisms}
\end{center}
\begin{center}
Examples are drawn from overly-technical domains

\nl

{\scriptsize
\begin{schemecode}
(define (deriv exp var)
  (cond ((number? exp) 0)
        ((variable? exp)
         (if (same-variable? exp var) 1 0))
        ((sum? exp)
         (make-sum (deriv (addend exp) var)
                   (deriv (augend exp) var)))
        ((product? exp)
         (make-sum
           (make-product (multiplier exp)
                         (deriv (multiplicand exp) var))
           (make-product (deriv (multiplier exp) var)
                         (multiplicand exp))))
        (else
         (error "unknown expression type -- DERIV" exp))))
\end{schemecode}
}
\end{center}
% TODO maybe improve this
\end{frame}


\begin{frame}

\begin{center}
{\bf Criticisms}
\end{center}
\begin{center}
Lacking coverage of foundational problem-solving techniques
\end{center}

\begin{block}{}
From an educational point of view, our experience suggests that undergraduate \\
computer science courses \highlight{should emphasize basic notions of modularity, specification, and}
\highlight{data abstraction}, and should not let these be displaced by more advanced topics,
such as design patterns, object-oriented methods, concurrency, functional languages,
and so on.

\nl

--- Jackson and Chapin, 2000
(emphasis mine)
\end{block}
\end{frame}


\imageslide[0.5]{htdp-cover.png}

\imageslide[0.55]{expected-four-args.png}

\imageslide[0.6]{language-levels.png}

\imageslide[0.5]{design-recipe.png}

\imageslide[0.6]{look-up-docs.png}

\imageslide[0.6]{incorrect-arguments.png}


% TODO paper screenshots about how you are sort of expected to learn programming along the way
